\documentclass[11pt]{article}

    \usepackage[breakable]{tcolorbox}
    \usepackage{parskip} % Stop auto-indenting (to mimic markdown behaviour)
    

    % Basic figure setup, for now with no caption control since it's done
    % automatically by Pandoc (which extracts ![](path) syntax from Markdown).
    \usepackage{graphicx}
    % Maintain compatibility with old templates. Remove in nbconvert 6.0
    \let\Oldincludegraphics\includegraphics
    % Ensure that by default, figures have no caption (until we provide a
    % proper Figure object with a Caption API and a way to capture that
    % in the conversion process - todo).
    \usepackage{caption}
    \DeclareCaptionFormat{nocaption}{}
    \captionsetup{format=nocaption,aboveskip=0pt,belowskip=0pt}

    \usepackage{float}
    \floatplacement{figure}{H} % forces figures to be placed at the correct location
    \usepackage{xcolor} % Allow colors to be defined
    \usepackage{enumerate} % Needed for markdown enumerations to work
    \usepackage{geometry} % Used to adjust the document margins
    \usepackage{amsmath} % Equations
    \usepackage{amssymb} % Equations
    \usepackage{textcomp} % defines textquotesingle
    % Hack from http://tex.stackexchange.com/a/47451/13684:
    \AtBeginDocument{%
        \def\PYZsq{\textquotesingle}% Upright quotes in Pygmentized code
    }
    \usepackage{upquote} % Upright quotes for verbatim code
    \usepackage{eurosym} % defines \euro

    \usepackage{iftex}
    \ifPDFTeX
        \usepackage[T1]{fontenc}
        \IfFileExists{alphabeta.sty}{
              \usepackage{alphabeta}
          }{
              \usepackage[mathletters]{ucs}
              \usepackage[utf8x]{inputenc}
          }
    \else
        \usepackage{fontspec}
        \usepackage{unicode-math}
    \fi

    \usepackage{fancyvrb} % verbatim replacement that allows latex
    \usepackage{grffile} % extends the file name processing of package graphics
                         % to support a larger range
    \makeatletter % fix for old versions of grffile with XeLaTeX
    \@ifpackagelater{grffile}{2019/11/01}
    {
      % Do nothing on new versions
    }
    {
      \def\Gread@@xetex#1{%
        \IfFileExists{"\Gin@base".bb}%
        {\Gread@eps{\Gin@base.bb}}%
        {\Gread@@xetex@aux#1}%
      }
    }
    \makeatother
    \usepackage[Export]{adjustbox} % Used to constrain images to a maximum size
    \adjustboxset{max size={0.9\linewidth}{0.9\paperheight}}

    % The hyperref package gives us a pdf with properly built
    % internal navigation ('pdf bookmarks' for the table of contents,
    % internal cross-reference links, web links for URLs, etc.)
    \usepackage{hyperref}
    % The default LaTeX title has an obnoxious amount of whitespace. By default,
    % titling removes some of it. It also provides customization options.
    \usepackage{titling}
    \usepackage{longtable} % longtable support required by pandoc >1.10
    \usepackage{booktabs}  % table support for pandoc > 1.12.2
    \usepackage{array}     % table support for pandoc >= 2.11.3
    \usepackage{calc}      % table minipage width calculation for pandoc >= 2.11.1
    \usepackage[inline]{enumitem} % IRkernel/repr support (it uses the enumerate* environment)
    \usepackage[normalem]{ulem} % ulem is needed to support strikethroughs (\sout)
                                % normalem makes italics be italics, not underlines
    \usepackage{mathrsfs}
    

    
    % Colors for the hyperref package
    \definecolor{urlcolor}{rgb}{0,.145,.698}
    \definecolor{linkcolor}{rgb}{.71,0.21,0.01}
    \definecolor{citecolor}{rgb}{.12,.54,.11}

    % ANSI colors
    \definecolor{ansi-black}{HTML}{3E424D}
    \definecolor{ansi-black-intense}{HTML}{282C36}
    \definecolor{ansi-red}{HTML}{E75C58}
    \definecolor{ansi-red-intense}{HTML}{B22B31}
    \definecolor{ansi-green}{HTML}{00A250}
    \definecolor{ansi-green-intense}{HTML}{007427}
    \definecolor{ansi-yellow}{HTML}{DDB62B}
    \definecolor{ansi-yellow-intense}{HTML}{B27D12}
    \definecolor{ansi-blue}{HTML}{208FFB}
    \definecolor{ansi-blue-intense}{HTML}{0065CA}
    \definecolor{ansi-magenta}{HTML}{D160C4}
    \definecolor{ansi-magenta-intense}{HTML}{A03196}
    \definecolor{ansi-cyan}{HTML}{60C6C8}
    \definecolor{ansi-cyan-intense}{HTML}{258F8F}
    \definecolor{ansi-white}{HTML}{C5C1B4}
    \definecolor{ansi-white-intense}{HTML}{A1A6B2}
    \definecolor{ansi-default-inverse-fg}{HTML}{FFFFFF}
    \definecolor{ansi-default-inverse-bg}{HTML}{000000}

    % common color for the border for error outputs.
    \definecolor{outerrorbackground}{HTML}{FFDFDF}

    % commands and environments needed by pandoc snippets
    % extracted from the output of `pandoc -s`
    \providecommand{\tightlist}{%
      \setlength{\itemsep}{0pt}\setlength{\parskip}{0pt}}
    \DefineVerbatimEnvironment{Highlighting}{Verbatim}{commandchars=\\\{\}}
    % Add ',fontsize=\small' for more characters per line
    \newenvironment{Shaded}{}{}
    \newcommand{\KeywordTok}[1]{\textcolor[rgb]{0.00,0.44,0.13}{\textbf{{#1}}}}
    \newcommand{\DataTypeTok}[1]{\textcolor[rgb]{0.56,0.13,0.00}{{#1}}}
    \newcommand{\DecValTok}[1]{\textcolor[rgb]{0.25,0.63,0.44}{{#1}}}
    \newcommand{\BaseNTok}[1]{\textcolor[rgb]{0.25,0.63,0.44}{{#1}}}
    \newcommand{\FloatTok}[1]{\textcolor[rgb]{0.25,0.63,0.44}{{#1}}}
    \newcommand{\CharTok}[1]{\textcolor[rgb]{0.25,0.44,0.63}{{#1}}}
    \newcommand{\StringTok}[1]{\textcolor[rgb]{0.25,0.44,0.63}{{#1}}}
    \newcommand{\CommentTok}[1]{\textcolor[rgb]{0.38,0.63,0.69}{\textit{{#1}}}}
    \newcommand{\OtherTok}[1]{\textcolor[rgb]{0.00,0.44,0.13}{{#1}}}
    \newcommand{\AlertTok}[1]{\textcolor[rgb]{1.00,0.00,0.00}{\textbf{{#1}}}}
    \newcommand{\FunctionTok}[1]{\textcolor[rgb]{0.02,0.16,0.49}{{#1}}}
    \newcommand{\RegionMarkerTok}[1]{{#1}}
    \newcommand{\ErrorTok}[1]{\textcolor[rgb]{1.00,0.00,0.00}{\textbf{{#1}}}}
    \newcommand{\NormalTok}[1]{{#1}}

    % Additional commands for more recent versions of Pandoc
    \newcommand{\ConstantTok}[1]{\textcolor[rgb]{0.53,0.00,0.00}{{#1}}}
    \newcommand{\SpecialCharTok}[1]{\textcolor[rgb]{0.25,0.44,0.63}{{#1}}}
    \newcommand{\VerbatimStringTok}[1]{\textcolor[rgb]{0.25,0.44,0.63}{{#1}}}
    \newcommand{\SpecialStringTok}[1]{\textcolor[rgb]{0.73,0.40,0.53}{{#1}}}
    \newcommand{\ImportTok}[1]{{#1}}
    \newcommand{\DocumentationTok}[1]{\textcolor[rgb]{0.73,0.13,0.13}{\textit{{#1}}}}
    \newcommand{\AnnotationTok}[1]{\textcolor[rgb]{0.38,0.63,0.69}{\textbf{\textit{{#1}}}}}
    \newcommand{\CommentVarTok}[1]{\textcolor[rgb]{0.38,0.63,0.69}{\textbf{\textit{{#1}}}}}
    \newcommand{\VariableTok}[1]{\textcolor[rgb]{0.10,0.09,0.49}{{#1}}}
    \newcommand{\ControlFlowTok}[1]{\textcolor[rgb]{0.00,0.44,0.13}{\textbf{{#1}}}}
    \newcommand{\OperatorTok}[1]{\textcolor[rgb]{0.40,0.40,0.40}{{#1}}}
    \newcommand{\BuiltInTok}[1]{{#1}}
    \newcommand{\ExtensionTok}[1]{{#1}}
    \newcommand{\PreprocessorTok}[1]{\textcolor[rgb]{0.74,0.48,0.00}{{#1}}}
    \newcommand{\AttributeTok}[1]{\textcolor[rgb]{0.49,0.56,0.16}{{#1}}}
    \newcommand{\InformationTok}[1]{\textcolor[rgb]{0.38,0.63,0.69}{\textbf{\textit{{#1}}}}}
    \newcommand{\WarningTok}[1]{\textcolor[rgb]{0.38,0.63,0.69}{\textbf{\textit{{#1}}}}}


    % Define a nice break command that doesn't care if a line doesn't already
    % exist.
    \def\br{\hspace*{\fill} \\* }
    % Math Jax compatibility definitions
    \def\gt{>}
    \def\lt{<}
    \let\Oldtex\TeX
    \let\Oldlatex\LaTeX
    \renewcommand{\TeX}{\textrm{\Oldtex}}
    \renewcommand{\LaTeX}{\textrm{\Oldlatex}}
    % Document parameters
    % Document title
    \title{ReportLovisaFinal}
    
    
    
    
    
% Pygments definitions
\makeatletter
\def\PY@reset{\let\PY@it=\relax \let\PY@bf=\relax%
    \let\PY@ul=\relax \let\PY@tc=\relax%
    \let\PY@bc=\relax \let\PY@ff=\relax}
\def\PY@tok#1{\csname PY@tok@#1\endcsname}
\def\PY@toks#1+{\ifx\relax#1\empty\else%
    \PY@tok{#1}\expandafter\PY@toks\fi}
\def\PY@do#1{\PY@bc{\PY@tc{\PY@ul{%
    \PY@it{\PY@bf{\PY@ff{#1}}}}}}}
\def\PY#1#2{\PY@reset\PY@toks#1+\relax+\PY@do{#2}}

\@namedef{PY@tok@w}{\def\PY@tc##1{\textcolor[rgb]{0.73,0.73,0.73}{##1}}}
\@namedef{PY@tok@c}{\let\PY@it=\textit\def\PY@tc##1{\textcolor[rgb]{0.24,0.48,0.48}{##1}}}
\@namedef{PY@tok@cp}{\def\PY@tc##1{\textcolor[rgb]{0.61,0.40,0.00}{##1}}}
\@namedef{PY@tok@k}{\let\PY@bf=\textbf\def\PY@tc##1{\textcolor[rgb]{0.00,0.50,0.00}{##1}}}
\@namedef{PY@tok@kp}{\def\PY@tc##1{\textcolor[rgb]{0.00,0.50,0.00}{##1}}}
\@namedef{PY@tok@kt}{\def\PY@tc##1{\textcolor[rgb]{0.69,0.00,0.25}{##1}}}
\@namedef{PY@tok@o}{\def\PY@tc##1{\textcolor[rgb]{0.40,0.40,0.40}{##1}}}
\@namedef{PY@tok@ow}{\let\PY@bf=\textbf\def\PY@tc##1{\textcolor[rgb]{0.67,0.13,1.00}{##1}}}
\@namedef{PY@tok@nb}{\def\PY@tc##1{\textcolor[rgb]{0.00,0.50,0.00}{##1}}}
\@namedef{PY@tok@nf}{\def\PY@tc##1{\textcolor[rgb]{0.00,0.00,1.00}{##1}}}
\@namedef{PY@tok@nc}{\let\PY@bf=\textbf\def\PY@tc##1{\textcolor[rgb]{0.00,0.00,1.00}{##1}}}
\@namedef{PY@tok@nn}{\let\PY@bf=\textbf\def\PY@tc##1{\textcolor[rgb]{0.00,0.00,1.00}{##1}}}
\@namedef{PY@tok@ne}{\let\PY@bf=\textbf\def\PY@tc##1{\textcolor[rgb]{0.80,0.25,0.22}{##1}}}
\@namedef{PY@tok@nv}{\def\PY@tc##1{\textcolor[rgb]{0.10,0.09,0.49}{##1}}}
\@namedef{PY@tok@no}{\def\PY@tc##1{\textcolor[rgb]{0.53,0.00,0.00}{##1}}}
\@namedef{PY@tok@nl}{\def\PY@tc##1{\textcolor[rgb]{0.46,0.46,0.00}{##1}}}
\@namedef{PY@tok@ni}{\let\PY@bf=\textbf\def\PY@tc##1{\textcolor[rgb]{0.44,0.44,0.44}{##1}}}
\@namedef{PY@tok@na}{\def\PY@tc##1{\textcolor[rgb]{0.41,0.47,0.13}{##1}}}
\@namedef{PY@tok@nt}{\let\PY@bf=\textbf\def\PY@tc##1{\textcolor[rgb]{0.00,0.50,0.00}{##1}}}
\@namedef{PY@tok@nd}{\def\PY@tc##1{\textcolor[rgb]{0.67,0.13,1.00}{##1}}}
\@namedef{PY@tok@s}{\def\PY@tc##1{\textcolor[rgb]{0.73,0.13,0.13}{##1}}}
\@namedef{PY@tok@sd}{\let\PY@it=\textit\def\PY@tc##1{\textcolor[rgb]{0.73,0.13,0.13}{##1}}}
\@namedef{PY@tok@si}{\let\PY@bf=\textbf\def\PY@tc##1{\textcolor[rgb]{0.64,0.35,0.47}{##1}}}
\@namedef{PY@tok@se}{\let\PY@bf=\textbf\def\PY@tc##1{\textcolor[rgb]{0.67,0.36,0.12}{##1}}}
\@namedef{PY@tok@sr}{\def\PY@tc##1{\textcolor[rgb]{0.64,0.35,0.47}{##1}}}
\@namedef{PY@tok@ss}{\def\PY@tc##1{\textcolor[rgb]{0.10,0.09,0.49}{##1}}}
\@namedef{PY@tok@sx}{\def\PY@tc##1{\textcolor[rgb]{0.00,0.50,0.00}{##1}}}
\@namedef{PY@tok@m}{\def\PY@tc##1{\textcolor[rgb]{0.40,0.40,0.40}{##1}}}
\@namedef{PY@tok@gh}{\let\PY@bf=\textbf\def\PY@tc##1{\textcolor[rgb]{0.00,0.00,0.50}{##1}}}
\@namedef{PY@tok@gu}{\let\PY@bf=\textbf\def\PY@tc##1{\textcolor[rgb]{0.50,0.00,0.50}{##1}}}
\@namedef{PY@tok@gd}{\def\PY@tc##1{\textcolor[rgb]{0.63,0.00,0.00}{##1}}}
\@namedef{PY@tok@gi}{\def\PY@tc##1{\textcolor[rgb]{0.00,0.52,0.00}{##1}}}
\@namedef{PY@tok@gr}{\def\PY@tc##1{\textcolor[rgb]{0.89,0.00,0.00}{##1}}}
\@namedef{PY@tok@ge}{\let\PY@it=\textit}
\@namedef{PY@tok@gs}{\let\PY@bf=\textbf}
\@namedef{PY@tok@gp}{\let\PY@bf=\textbf\def\PY@tc##1{\textcolor[rgb]{0.00,0.00,0.50}{##1}}}
\@namedef{PY@tok@go}{\def\PY@tc##1{\textcolor[rgb]{0.44,0.44,0.44}{##1}}}
\@namedef{PY@tok@gt}{\def\PY@tc##1{\textcolor[rgb]{0.00,0.27,0.87}{##1}}}
\@namedef{PY@tok@err}{\def\PY@bc##1{{\setlength{\fboxsep}{\string -\fboxrule}\fcolorbox[rgb]{1.00,0.00,0.00}{1,1,1}{\strut ##1}}}}
\@namedef{PY@tok@kc}{\let\PY@bf=\textbf\def\PY@tc##1{\textcolor[rgb]{0.00,0.50,0.00}{##1}}}
\@namedef{PY@tok@kd}{\let\PY@bf=\textbf\def\PY@tc##1{\textcolor[rgb]{0.00,0.50,0.00}{##1}}}
\@namedef{PY@tok@kn}{\let\PY@bf=\textbf\def\PY@tc##1{\textcolor[rgb]{0.00,0.50,0.00}{##1}}}
\@namedef{PY@tok@kr}{\let\PY@bf=\textbf\def\PY@tc##1{\textcolor[rgb]{0.00,0.50,0.00}{##1}}}
\@namedef{PY@tok@bp}{\def\PY@tc##1{\textcolor[rgb]{0.00,0.50,0.00}{##1}}}
\@namedef{PY@tok@fm}{\def\PY@tc##1{\textcolor[rgb]{0.00,0.00,1.00}{##1}}}
\@namedef{PY@tok@vc}{\def\PY@tc##1{\textcolor[rgb]{0.10,0.09,0.49}{##1}}}
\@namedef{PY@tok@vg}{\def\PY@tc##1{\textcolor[rgb]{0.10,0.09,0.49}{##1}}}
\@namedef{PY@tok@vi}{\def\PY@tc##1{\textcolor[rgb]{0.10,0.09,0.49}{##1}}}
\@namedef{PY@tok@vm}{\def\PY@tc##1{\textcolor[rgb]{0.10,0.09,0.49}{##1}}}
\@namedef{PY@tok@sa}{\def\PY@tc##1{\textcolor[rgb]{0.73,0.13,0.13}{##1}}}
\@namedef{PY@tok@sb}{\def\PY@tc##1{\textcolor[rgb]{0.73,0.13,0.13}{##1}}}
\@namedef{PY@tok@sc}{\def\PY@tc##1{\textcolor[rgb]{0.73,0.13,0.13}{##1}}}
\@namedef{PY@tok@dl}{\def\PY@tc##1{\textcolor[rgb]{0.73,0.13,0.13}{##1}}}
\@namedef{PY@tok@s2}{\def\PY@tc##1{\textcolor[rgb]{0.73,0.13,0.13}{##1}}}
\@namedef{PY@tok@sh}{\def\PY@tc##1{\textcolor[rgb]{0.73,0.13,0.13}{##1}}}
\@namedef{PY@tok@s1}{\def\PY@tc##1{\textcolor[rgb]{0.73,0.13,0.13}{##1}}}
\@namedef{PY@tok@mb}{\def\PY@tc##1{\textcolor[rgb]{0.40,0.40,0.40}{##1}}}
\@namedef{PY@tok@mf}{\def\PY@tc##1{\textcolor[rgb]{0.40,0.40,0.40}{##1}}}
\@namedef{PY@tok@mh}{\def\PY@tc##1{\textcolor[rgb]{0.40,0.40,0.40}{##1}}}
\@namedef{PY@tok@mi}{\def\PY@tc##1{\textcolor[rgb]{0.40,0.40,0.40}{##1}}}
\@namedef{PY@tok@il}{\def\PY@tc##1{\textcolor[rgb]{0.40,0.40,0.40}{##1}}}
\@namedef{PY@tok@mo}{\def\PY@tc##1{\textcolor[rgb]{0.40,0.40,0.40}{##1}}}
\@namedef{PY@tok@ch}{\let\PY@it=\textit\def\PY@tc##1{\textcolor[rgb]{0.24,0.48,0.48}{##1}}}
\@namedef{PY@tok@cm}{\let\PY@it=\textit\def\PY@tc##1{\textcolor[rgb]{0.24,0.48,0.48}{##1}}}
\@namedef{PY@tok@cpf}{\let\PY@it=\textit\def\PY@tc##1{\textcolor[rgb]{0.24,0.48,0.48}{##1}}}
\@namedef{PY@tok@c1}{\let\PY@it=\textit\def\PY@tc##1{\textcolor[rgb]{0.24,0.48,0.48}{##1}}}
\@namedef{PY@tok@cs}{\let\PY@it=\textit\def\PY@tc##1{\textcolor[rgb]{0.24,0.48,0.48}{##1}}}

\def\PYZbs{\char`\\}
\def\PYZus{\char`\_}
\def\PYZob{\char`\{}
\def\PYZcb{\char`\}}
\def\PYZca{\char`\^}
\def\PYZam{\char`\&}
\def\PYZlt{\char`\<}
\def\PYZgt{\char`\>}
\def\PYZsh{\char`\#}
\def\PYZpc{\char`\%}
\def\PYZdl{\char`\$}
\def\PYZhy{\char`\-}
\def\PYZsq{\char`\'}
\def\PYZdq{\char`\"}
\def\PYZti{\char`\~}
% for compatibility with earlier versions
\def\PYZat{@}
\def\PYZlb{[}
\def\PYZrb{]}
\makeatother


    % For linebreaks inside Verbatim environment from package fancyvrb.
    \makeatletter
        \newbox\Wrappedcontinuationbox
        \newbox\Wrappedvisiblespacebox
        \newcommand*\Wrappedvisiblespace {\textcolor{red}{\textvisiblespace}}
        \newcommand*\Wrappedcontinuationsymbol {\textcolor{red}{\llap{\tiny$\m@th\hookrightarrow$}}}
        \newcommand*\Wrappedcontinuationindent {3ex }
        \newcommand*\Wrappedafterbreak {\kern\Wrappedcontinuationindent\copy\Wrappedcontinuationbox}
        % Take advantage of the already applied Pygments mark-up to insert
        % potential linebreaks for TeX processing.
        %        {, <, #, %, $, ' and ": go to next line.
        %        _, }, ^, &, >, - and ~: stay at end of broken line.
        % Use of \textquotesingle for straight quote.
        \newcommand*\Wrappedbreaksatspecials {%
            \def\PYGZus{\discretionary{\char`\_}{\Wrappedafterbreak}{\char`\_}}%
            \def\PYGZob{\discretionary{}{\Wrappedafterbreak\char`\{}{\char`\{}}%
            \def\PYGZcb{\discretionary{\char`\}}{\Wrappedafterbreak}{\char`\}}}%
            \def\PYGZca{\discretionary{\char`\^}{\Wrappedafterbreak}{\char`\^}}%
            \def\PYGZam{\discretionary{\char`\&}{\Wrappedafterbreak}{\char`\&}}%
            \def\PYGZlt{\discretionary{}{\Wrappedafterbreak\char`\<}{\char`\<}}%
            \def\PYGZgt{\discretionary{\char`\>}{\Wrappedafterbreak}{\char`\>}}%
            \def\PYGZsh{\discretionary{}{\Wrappedafterbreak\char`\#}{\char`\#}}%
            \def\PYGZpc{\discretionary{}{\Wrappedafterbreak\char`\%}{\char`\%}}%
            \def\PYGZdl{\discretionary{}{\Wrappedafterbreak\char`\$}{\char`\$}}%
            \def\PYGZhy{\discretionary{\char`\-}{\Wrappedafterbreak}{\char`\-}}%
            \def\PYGZsq{\discretionary{}{\Wrappedafterbreak\textquotesingle}{\textquotesingle}}%
            \def\PYGZdq{\discretionary{}{\Wrappedafterbreak\char`\"}{\char`\"}}%
            \def\PYGZti{\discretionary{\char`\~}{\Wrappedafterbreak}{\char`\~}}%
        }
        % Some characters . , ; ? ! / are not pygmentized.
        % This macro makes them "active" and they will insert potential linebreaks
        \newcommand*\Wrappedbreaksatpunct {%
            \lccode`\~`\.\lowercase{\def~}{\discretionary{\hbox{\char`\.}}{\Wrappedafterbreak}{\hbox{\char`\.}}}%
            \lccode`\~`\,\lowercase{\def~}{\discretionary{\hbox{\char`\,}}{\Wrappedafterbreak}{\hbox{\char`\,}}}%
            \lccode`\~`\;\lowercase{\def~}{\discretionary{\hbox{\char`\;}}{\Wrappedafterbreak}{\hbox{\char`\;}}}%
            \lccode`\~`\:\lowercase{\def~}{\discretionary{\hbox{\char`\:}}{\Wrappedafterbreak}{\hbox{\char`\:}}}%
            \lccode`\~`\?\lowercase{\def~}{\discretionary{\hbox{\char`\?}}{\Wrappedafterbreak}{\hbox{\char`\?}}}%
            \lccode`\~`\!\lowercase{\def~}{\discretionary{\hbox{\char`\!}}{\Wrappedafterbreak}{\hbox{\char`\!}}}%
            \lccode`\~`\/\lowercase{\def~}{\discretionary{\hbox{\char`\/}}{\Wrappedafterbreak}{\hbox{\char`\/}}}%
            \catcode`\.\active
            \catcode`\,\active
            \catcode`\;\active
            \catcode`\:\active
            \catcode`\?\active
            \catcode`\!\active
            \catcode`\/\active
            \lccode`\~`\~
        }
    \makeatother

    \let\OriginalVerbatim=\Verbatim
    \makeatletter
    \renewcommand{\Verbatim}[1][1]{%
        %\parskip\z@skip
        \sbox\Wrappedcontinuationbox {\Wrappedcontinuationsymbol}%
        \sbox\Wrappedvisiblespacebox {\FV@SetupFont\Wrappedvisiblespace}%
        \def\FancyVerbFormatLine ##1{\hsize\linewidth
            \vtop{\raggedright\hyphenpenalty\z@\exhyphenpenalty\z@
                \doublehyphendemerits\z@\finalhyphendemerits\z@
                \strut ##1\strut}%
        }%
        % If the linebreak is at a space, the latter will be displayed as visible
        % space at end of first line, and a continuation symbol starts next line.
        % Stretch/shrink are however usually zero for typewriter font.
        \def\FV@Space {%
            \nobreak\hskip\z@ plus\fontdimen3\font minus\fontdimen4\font
            \discretionary{\copy\Wrappedvisiblespacebox}{\Wrappedafterbreak}
            {\kern\fontdimen2\font}%
        }%

        % Allow breaks at special characters using \PYG... macros.
        \Wrappedbreaksatspecials
        % Breaks at punctuation characters . , ; ? ! and / need catcode=\active
        \OriginalVerbatim[#1,codes*=\Wrappedbreaksatpunct]%
    }
    \makeatother

    % Exact colors from NB
    \definecolor{incolor}{HTML}{303F9F}
    \definecolor{outcolor}{HTML}{D84315}
    \definecolor{cellborder}{HTML}{CFCFCF}
    \definecolor{cellbackground}{HTML}{F7F7F7}

    % prompt
    \makeatletter
    \newcommand{\boxspacing}{\kern\kvtcb@left@rule\kern\kvtcb@boxsep}
    \makeatother
    \newcommand{\prompt}[4]{
        {\ttfamily\llap{{\color{#2}[#3]:\hspace{3pt}#4}}\vspace{-\baselineskip}}
    }
    

    
    % Prevent overflowing lines due to hard-to-break entities
    \sloppy
    % Setup hyperref package
    \hypersetup{
      breaklinks=true,  % so long urls are correctly broken across lines
      colorlinks=true,
      urlcolor=urlcolor,
      linkcolor=linkcolor,
      citecolor=citecolor,
      }
    % Slightly bigger margins than the latex defaults
    
    \geometry{verbose,tmargin=1in,bmargin=1in,lmargin=1in,rmargin=1in}
    
    

\begin{document}
    
    \maketitle
    
    

    
    \hypertarget{abstract}{%
\section{Abstract}\label{abstract}}

Arctic new particle formation is an import source of cloud condensation
nuclei which in turn affect cloud properties. Insights into the drivers
of new particle formation in a changing climate are important, as clouds
in the Arctic contributes to further warming of an already vulnerable
environment. Sea ice decline has been observed to be correlated with
increasing new particle formation at the Zeppelin Observatory, but the
underlying processes are complex, and the question is whether this hold
true for more recent years. In this work, DMPS and CPC data from
Zeppelin between 2010-2020 as well as satellite data on sea ice and
chlorophyll concentration. The correlation between new particle
formation and sea ice is investigated and different methods are tested
to identify a suitable indicator for detection of new particle formation
in the size distribution data. K-means cluster analysis was chosen to
best represent the which partitions the data into \(k\) clusters based
on the shape of the size distribution. The results suggest that there is
no correlation between the annual new particle formation frequency and
sea ice concentration in the Greenland and Barents Sea for the years
2010-2017, which differs from previous observations at Zeppelin between
2000-2010. Further investigations targeting for example air mass
trajectory analysis and suitable new particle formation indicators are
needed to obtain amore complete understanding of the complex interplay
between new particle formation drivers.

    \hypertarget{acknowledgements}{%
\section{Acknowledgements}\label{acknowledgements}}

First and foremost, I would like to acknowledge the providers of the
data sets OSI-SAF consortium (sea ice data) and OBS-ESACCI-OC for
chlorophyll data, and the Norwegian Polar Institute (NPI) for running
the Zeppelin station and EBAS for managing the database.

I collaborated with Kei Tsuruhara and our supervisor Dominic
Heslin-Reese. I am very thankful to be part of this team, it has been
inspiring (and fun!) to work with them. In this report Kei has produced
the sea ice data (it was not easy to work with that dataset!), Dominic
provided the DMPS and CPC data as well as some functions to read in and
do the basic preparation of data. Other than I have received tremendous
amount of help while learning Python (and data analysis in general) from
especially Dominic but also from other course instructors. Thanks Paul
Zieger for enthusiasm and great advices on things to look more into.
Thanks to Linn Karlsson whose work in a previous version of this course
happened to include a great function to integrate log-normal
distributions.

    \hypertarget{introduction}{%
\section{Introduction}\label{introduction}}

The Arctic has experienced a much faster warming than other part of the
globe. The response to increased temperature in the Arctic environment
is manyfold and implies drastic changes to of the whole environment and
ecosystems. Drastic reductions in sea ice cover (Perovich et al., 2018)
occurs oceans are becoming more productive which implies increased
emissions new particle formation precursor gases (Galí et al.~2019). New
particle formation (NPF) play an important role in the Arctic radiation
budget as the may grow to cloud condensation nuclei's (CCN:s) and
subsequently activate into cloud droplets. Clouds exert a net warming
effect in the Arctic, which makes studies of NPF drivers important.

The Arctic is in many ways an extreme environment which is also
reflected in the annual cycle of airborne particulate matter (Tunved et
al.~2013). During the dark months, ocean and air circulation patterns
implies an influx of anthropogenic pollutants, and the aerosol size
distribution is dominated by accumulation mode particles. In combination
with the Arctic wintertime meteorological conditions, this leads to the
formation of a persistent Arctic haze. As the sun gradually returns, the
circulation and airmass transport patterns changes and the anthropogenic
pollution inflow is restricted. Formation of low-level clouds during
summertime leads to removal of accumulation mode particles by wet
scavenging marks the end of the Arctic Haze period. Consequently, the
composition, properties, and size distribution of Arctic aerosol
particle change. The lower aerosol load and increased availability of
sunlight promotes photochemistry and results in a peak of in aerosol
particles smaller than 100 nm (Schmale \& Baccarini, 2021).

At the Zeppelin mountain station on Svalbard long-term measurements are
conducted to monitor the properties of aerosol particles. Long term
measurements are valuable to monitor changes of the sensitive Arctic
environment. Data from eleven years of DMPS measurements (2000-2010) at
the Zeppelin station have been used to present the hypothesis that
Arctic sea ice reduction leads to increased NPF frequency at Zeppelin
(Dall'Osto et al.~2017). The question this project makes an attempt to
answer is weather this is a causal effect, if this hold true for
subsequent years and if trends in proxies for productivity of oceans
might be driving NPF as well. Another aim is to partly reproduce the
Dall'Osto paper for a later time period by applying similar methodology
with regards to NPF identification in the DMPS data, as well as to
investigate other possibilities for NPF identification. We will use
satellite data on sea ice and chlorophyll, and aerosol size distribution
and number concentration data from Zeppelin.

    \hypertarget{method}{%
\section{Method}\label{method}}

\hypertarget{datsets}{%
\subsection{Datsets}\label{datsets}}

We have used level 2 Differential Mobility Particle Sizer (DMPS) and
data from two different Condensation particle counters (CPC's) from the
Arctic Zeppelin mountain station at Svalbard, provided by our supervisor
Dominic Heslin-Rees, and satellite retrieved sea ice concentration from
OSI-SAF and satellite chlorophyll observations OBS-ESACCI-OC. The DMPS
and CPC data encompass the years 2010-2020 with overall high data
coverage. The data was supplied in 30 min resolution. The satellite data
was processed Kei and DMPS and CPC data is the focus of this report.

The DMPS measures the particle number size distributions for particles
smaller than 1000 nm. The DMPS scans over different sizes and measures
the number concentration (\#/cm\(^3\)) in each size interval, or bin.
The output is in the form of log normal particle concentration which
accounts for the difference bin bin width. (CPC's) simply counts
particles in the sub micrometer range and provides the particle number
concentration (\#/cm\(^3\)).

    \hypertarget{import-packages}{%
\subsection{Import packages}\label{import-packages}}

And make sure that functions are auto-updated.

    \begin{tcolorbox}[breakable, size=fbox, boxrule=1pt, pad at break*=1mm,colback=cellbackground, colframe=cellborder]
\prompt{In}{incolor}{1}{\boxspacing}
\begin{Verbatim}[commandchars=\\\{\}]
\PY{k+kn}{import} \PY{n+nn}{DMPS\PYZus{}functions} \PY{k}{as} \PY{n+nn}{fu}
\PY{k+kn}{import} \PY{n+nn}{numpy} \PY{k}{as} \PY{n+nn}{np}
\PY{k+kn}{import} \PY{n+nn}{pandas} \PY{k}{as} \PY{n+nn}{pd}
\PY{k+kn}{import} \PY{n+nn}{matplotlib}\PY{n+nn}{.}\PY{n+nn}{pyplot} \PY{k}{as} \PY{n+nn}{plt}
\PY{k+kn}{import} \PY{n+nn}{datetime} 
\PY{k+kn}{import} \PY{n+nn}{seaborn} \PY{k}{as} \PY{n+nn}{sns}
\PY{k+kn}{import} \PY{n+nn}{scipy} \PY{k}{as} \PY{n+nn}{sc}
\PY{k+kn}{import} \PY{n+nn}{glob} 
\PY{k+kn}{import} \PY{n+nn}{sklearn}\PY{n+nn}{.}\PY{n+nn}{metrics} \PY{k}{as} \PY{n+nn}{skm}
\PY{k+kn}{from} \PY{n+nn}{sklearn} \PY{k+kn}{import} \PY{n}{metrics}
\PY{k+kn}{from} \PY{n+nn}{sklearn}\PY{n+nn}{.}\PY{n+nn}{metrics}\PY{n+nn}{.}\PY{n+nn}{pairwise} \PY{k+kn}{import} \PY{n}{pairwise\PYZus{}distances}
\PY{k+kn}{from} \PY{n+nn}{sklearn}\PY{n+nn}{.}\PY{n+nn}{cluster} \PY{k+kn}{import} \PY{n}{KMeans}
\PY{k+kn}{import} \PY{n+nn}{scipy}\PY{n+nn}{.}\PY{n+nn}{stats} \PY{k}{as} \PY{n+nn}{st}
\PY{k+kn}{import} \PY{n+nn}{seaborn} \PY{k}{as} \PY{n+nn}{sns}
\PY{k+kn}{from} \PY{n+nn}{matplotlib} \PY{k+kn}{import} \PY{n}{cm}

\PY{c+c1}{\PYZsh{} Auto\PYZhy{}update the functions. Useful if the functions have been altered. }
\PY{o}{\PYZpc{}}\PY{n}{load\PYZus{}ext} \PY{n}{autoreload}
\PY{o}{\PYZpc{}}\PY{n}{autoreload} \PY{l+m+mi}{2}

\PY{n}{plt}\PY{o}{.}\PY{n}{rcParams}\PY{p}{[}\PY{l+s+s1}{\PYZsq{}}\PY{l+s+s1}{font.size}\PY{l+s+s1}{\PYZsq{}}\PY{p}{]} \PY{o}{=} \PY{l+s+s1}{\PYZsq{}}\PY{l+s+s1}{12}\PY{l+s+s1}{\PYZsq{}}
\end{Verbatim}
\end{tcolorbox}

    \begin{tcolorbox}[breakable, size=fbox, boxrule=1pt, pad at break*=1mm,colback=cellbackground, colframe=cellborder]
\prompt{In}{incolor}{2}{\boxspacing}
\begin{Verbatim}[commandchars=\\\{\}]
\PY{c+c1}{\PYZsh{} Defining the path containg all DMPS data}
\PY{n}{path} \PY{o}{=} \PY{l+s+s1}{\PYZsq{}}\PY{l+s+s1}{C:}\PY{l+s+se}{\PYZbs{}\PYZbs{}}\PY{l+s+s1}{Users}\PY{l+s+se}{\PYZbs{}\PYZbs{}}\PY{l+s+s1}{Lovisa}\PY{l+s+se}{\PYZbs{}\PYZbs{}}\PY{l+s+s1}{Documents}\PY{l+s+se}{\PYZbs{}\PYZbs{}}\PY{l+s+s1}{Courses II}\PY{l+s+se}{\PYZbs{}\PYZbs{}}\PY{l+s+s1}{E\PYZus{}science\PYZus{}tools HT22}\PY{l+s+se}{\PYZbs{}\PYZbs{}}\PY{l+s+s1}{Project}\PY{l+s+se}{\PYZbs{}\PYZbs{}}\PY{l+s+s1}{DATA\PYZus{}ZEP\PYZus{}2010to2020}\PY{l+s+se}{\PYZbs{}\PYZbs{}}\PY{l+s+s1}{\PYZsq{}}
\end{Verbatim}
\end{tcolorbox}

    \begin{tcolorbox}[breakable, size=fbox, boxrule=1pt, pad at break*=1mm,colback=cellbackground, colframe=cellborder]
\prompt{In}{incolor}{3}{\boxspacing}
\begin{Verbatim}[commandchars=\\\{\}]
\PY{c+c1}{\PYZsh{} Using glob to import the files in the specified path}
\PY{n}{fu}\PY{o}{.}\PY{n}{importData}\PY{p}{(}\PY{n}{path}\PY{p}{)}
\end{Verbatim}
\end{tcolorbox}

    Loading each year of 30 min resolution DMPS data into a dataframe and
create a list containing all dataframes. Column headings are added, and
data flagged as invalid is removed.

    \begin{tcolorbox}[breakable, size=fbox, boxrule=1pt, pad at break*=1mm,colback=cellbackground, colframe=cellborder]
\prompt{In}{incolor}{4}{\boxspacing}
\begin{Verbatim}[commandchars=\\\{\}]
\PY{n}{DFs\PYZus{}DMPS} \PY{o}{=} \PY{n}{fu}\PY{o}{.}\PY{n}{load\PYZus{}and\PYZus{}append\PYZus{}DMPS}\PY{p}{(}\PY{n}{path}\PY{p}{,} 
                             \PY{n}{name\PYZus{}in\PYZus{}file}\PY{o}{=}\PY{l+s+s1}{\PYZsq{}}\PY{l+s+s1}{DMPS}\PY{l+s+s1}{\PYZsq{}}\PY{p}{)}
\end{Verbatim}
\end{tcolorbox}

    \hypertarget{concatenate-dmps-data}{%
\subsection{Concatenate DMPS data}\label{concatenate-dmps-data}}

The DMPS data comes in several files and must be concatenated into a
common dateframe, and unessary columns are dropped.

    \begin{tcolorbox}[breakable, size=fbox, boxrule=1pt, pad at break*=1mm,colback=cellbackground, colframe=cellborder]
\prompt{In}{incolor}{5}{\boxspacing}
\begin{Verbatim}[commandchars=\\\{\}]
\PY{n}{df\PYZus{}DMPS} \PY{o}{=} \PY{n}{fu}\PY{o}{.}\PY{n}{concat\PYZus{}df\PYZus{}DMPS}\PY{p}{(}\PY{n}{DFs\PYZus{}DMPS}\PY{p}{)}
\end{Verbatim}
\end{tcolorbox}

    Useful variables are defined column headings renamed to ease calling of
specific columns.

    \begin{tcolorbox}[breakable, size=fbox, boxrule=1pt, pad at break*=1mm,colback=cellbackground, colframe=cellborder]
\prompt{In}{incolor}{6}{\boxspacing}
\begin{Verbatim}[commandchars=\\\{\}]
\PY{c+c1}{\PYZsh{} Get a list with the DMPS bin column names as strings and floats to call column in dataframe and for calculations.}
\PY{n}{bin\PYZus{}col\PYZus{}list} \PY{o}{=} \PY{n}{fu}\PY{o}{.}\PY{n}{get\PYZus{}bin\PYZus{}column\PYZus{}string\PYZus{}list}\PY{p}{(}\PY{p}{)}
\PY{n}{bin\PYZus{}cols} \PY{o}{=} \PY{n}{fu}\PY{o}{.}\PY{n}{get\PYZus{}bins}\PY{p}{(}\PY{n}{bin\PYZus{}col\PYZus{}list}\PY{p}{)}
\PY{n}{bin\PYZus{}cols} \PY{o}{=} \PY{p}{[}\PY{n+nb}{float}\PY{p}{(}\PY{n}{x}\PY{p}{)} \PY{k}{for} \PY{n}{x} \PY{o+ow}{in} \PY{n}{bin\PYZus{}cols}\PY{p}{]}

\PY{c+c1}{\PYZsh{} Rename the column headings for midpoint diameters to numeric values with less decimals.}
\PY{n}{df\PYZus{}DMPS} \PY{o}{=} \PY{n}{fu}\PY{o}{.}\PY{n}{renameDpColumns}\PY{p}{(}\PY{n}{df\PYZus{}DMPS}\PY{p}{,} \PY{n}{bin\PYZus{}col\PYZus{}list}\PY{p}{)}

\PY{c+c1}{\PYZsh{} Make a list of all the \PYZdq{}real\PYZdq{} non\PYZhy{}rounded midpoint diameters as a list and as array.}
\PY{n}{diameterList}\PY{p}{,} \PY{n}{diameters} \PY{o}{=} \PY{n}{fu}\PY{o}{.}\PY{n}{getFloatDiameterListAndArray}\PY{p}{(}\PY{p}{)}

\PY{c+c1}{\PYZsh{} Make a list of diameters as string\PYZhy{}objects}
\PY{n}{diameters\PYZus{}as\PYZus{}strings} \PY{o}{=} \PY{p}{[}\PY{n+nb}{str}\PY{p}{(}\PY{n}{x}\PY{p}{)} \PY{k}{for} \PY{n}{x} \PY{o+ow}{in} \PY{n}{diameters}\PY{p}{]}
\end{Verbatim}
\end{tcolorbox}

    \hypertarget{resampling-to-hourly-averages}{%
\subsection{Resampling to hourly
averages}\label{resampling-to-hourly-averages}}

The function \texttt{resample} was used to convert the 30 min DMPS
particle size distribution data to a dataframe containing the 1 hour
average. The python function \texttt{resample} fills the gaps in the
data by adding rows for gap days/months/hours with NaN-values to get a
continous dataset. Because of this, rows containing only NaN values are
dropped.

    \begin{tcolorbox}[breakable, size=fbox, boxrule=1pt, pad at break*=1mm,colback=cellbackground, colframe=cellborder]
\prompt{In}{incolor}{7}{\boxspacing}
\begin{Verbatim}[commandchars=\\\{\}]
\PY{n}{df\PYZus{}hourly\PYZus{}2010\PYZus{}2020\PYZus{}mean}  \PY{o}{=} \PY{n}{df\PYZus{}DMPS}\PY{o}{.}\PY{n}{resample}\PY{p}{(}\PY{l+s+s1}{\PYZsq{}}\PY{l+s+s1}{H}\PY{l+s+s1}{\PYZsq{}}\PY{p}{)}\PY{o}{.}\PY{n}{mean}\PY{p}{(}\PY{p}{)}
\PY{n}{df\PYZus{}hourly\PYZus{}2010\PYZus{}2020\PYZus{}mean}  \PY{o}{=} \PY{n}{df\PYZus{}hourly\PYZus{}2010\PYZus{}2020\PYZus{}mean}\PY{o}{.}\PY{n}{dropna}\PY{p}{(}\PY{n}{how}\PY{o}{=}\PY{l+s+s1}{\PYZsq{}}\PY{l+s+s1}{all}\PY{l+s+s1}{\PYZsq{}}\PY{p}{)}
\end{Verbatim}
\end{tcolorbox}

    The first two columns in the dataframe contain data from two different
CPC:s, i.e.~condensation particle counters which count particles and
yield the particle concentration in \#/cm\(^3\). The UFCPC (ultrafine
CPC) has a lower particle diameter threshold at 3 nm in comparision with
the CPC3010 whose lower threshold is at 10 nm. The third colum Ntot
contains the total particles number concentration as computed by the
DMPS. The rest of the columns specify the midpoint size (nm) of the bins
for which the values indicate the log-normal distribution (\#/cm\(^3\)).
The last column (not visable indicates flags indicating the validity of
the data. Unvalid data was removed by \texttt{concat\_df\_DMPS}
function.

    \hypertarget{identifying-npf-events}{%
\subsection{Identifying NPF events}\label{identifying-npf-events}}

One aims of this report is to conduct a brief investigation of different
methods to identify NPF events. Some different methods are tested to
identify the NPF events:

\begin{itemize}
\tightlist
\item
  By calculation of \(\frac{N_{D_p<10nm}}{N_{tot}}\)
\item
  By using K-means clustering
\item
  By calculating the difference between the UFCPC and the CPC
\end{itemize}

Below the methods are decribed in more detail.

    \hypertarget{method-1-using-n_d_p-xn_tot}{%
\subsubsection{\texorpdfstring{Method 1: Using
\(N_{D_P <x}/N_{tot}\)}{Method 1: Using N\_\{D\_P \textless x\}/N\_\{tot\}}}\label{method-1-using-n_d_p-xn_tot}}

The motivation to use the first metod is that in the event of NPF, there
will by a drastic increase in number concentration of particles in
smallest size bins, and the ratio of particles in the smaller bins to
should be elevated. The first method requires that the total number
concentration \(N_{tot}\) is calculated in a given diameter range. To do
this, the log-normal distribution which is provided by the DMPS has to
be integrated. Linn Karlsson (e-Science course participant 2017) should
be acknowledged for great inspiration to the method used in this work
(Linn's method invloved the use of dictionaries but here arrays are used
instead).

The \texttt{calcNtot} calculates the particle concentration by
integrating in a given bin midpoint diameter interval \texttt{diameters}
and adds a column in the returned dataframe containg the calculated
values.

    \begin{tcolorbox}[breakable, size=fbox, boxrule=1pt, pad at break*=1mm,colback=cellbackground, colframe=cellborder]
\prompt{In}{incolor}{8}{\boxspacing}
\begin{Verbatim}[commandchars=\\\{\}]
\PY{c+c1}{\PYZsh{} Calculate the total particle number concentration for the 1 h resolution data}
\PY{n}{df\PYZus{}hourly\PYZus{}2010\PYZus{}2020\PYZus{}mean\PYZus{}ntotCalc}  \PY{o}{=} \PY{n}{fu}\PY{o}{.}\PY{n}{calcNtot}\PY{p}{(}\PY{n}{diameters}\PY{p}{,} \PY{n}{df\PYZus{}hourly\PYZus{}2010\PYZus{}2020\PYZus{}mean}\PY{p}{,}\PY{n}{diameters\PYZus{}as\PYZus{}strings}\PY{p}{)}
\end{Verbatim}
\end{tcolorbox}

    In Figure 1, the result from the integration of the full size
distribution is compared to Ntot given in the DMPS data, and they appear
to be in good aggrement (slope of 1 and \(R^2\) close to 1).

    \begin{tcolorbox}[breakable, size=fbox, boxrule=1pt, pad at break*=1mm,colback=cellbackground, colframe=cellborder]
\prompt{In}{incolor}{9}{\boxspacing}
\begin{Verbatim}[commandchars=\\\{\}]
\PY{n}{varx} \PY{o}{=} \PY{n}{df\PYZus{}hourly\PYZus{}2010\PYZus{}2020\PYZus{}mean\PYZus{}ntotCalc}\PY{p}{[}\PY{l+s+s1}{\PYZsq{}}\PY{l+s+s1}{NtotCalc}\PY{l+s+s1}{\PYZsq{}}\PY{p}{]}\PY{o}{.}\PY{n}{values}
\PY{n}{vary} \PY{o}{=} \PY{n}{df\PYZus{}hourly\PYZus{}2010\PYZus{}2020\PYZus{}mean\PYZus{}ntotCalc}\PY{p}{[}\PY{l+s+s1}{\PYZsq{}}\PY{l+s+s1}{Ntot}\PY{l+s+s1}{\PYZsq{}}\PY{p}{]}\PY{o}{.}\PY{n}{values}

\PY{n}{fu}\PY{o}{.}\PY{n}{compareIntegration}\PY{p}{(}\PY{n}{varx}\PY{p}{,}\PY{n}{vary}\PY{p}{)}
\end{Verbatim}
\end{tcolorbox}

    \begin{Verbatim}[commandchars=\\\{\}]
R-squared: 0.999363
Intercept: 0.0445901316076629
Slope: 1.0004705430431775
    \end{Verbatim}

    \begin{center}
    \adjustimage{max size={0.9\linewidth}{0.9\paperheight}}{ReportLovisaFinal_files/ReportLovisaFinal_21_1.png}
    \end{center}
    { \hspace*{\fill} \\}
    
    \textbf{Figure 1. Comparison of the integrated total particle number
concentration and the total particle number concentration given by the
DMPS.}

    \hypertarget{method-2-k-means-clustering}{%
\subsubsection{Method 2: K-means
clustering}\label{method-2-k-means-clustering}}

    The second approach to identify NPF events involves K-means clustering
(using \texttt{sklearn.cluster.Kmeans}) of the long-term DMPS dataset
which is an unsupervised machine learning algorithm. In short,
unsupervised machine learning as opposed to supervised, involves data
without labels and the data is grouped based in similarity of the data.
This approach was previously deployed by Dall'Osto et al.~(2017) to
identify NPF events from size distribution data at the Zeppelin, and
therefore it is interesting try the same approach. Clustering of data
implies grouping similar objects together based on different features of
the data. The K-means algorithm means that the data is divided into to a
user defined number of cluster (\(k\)). The K-means algorithm starts by
picking k random datapoints and assigns them as centers of the cluster
(centroids). Then distance of the other data points to the centroids are
then calculated and they are assigned to the nearest cluster centroid.
Then the process of assigning new centroids starts over and the data is
reassigned. This process is then repeated several times until the final
clusters are established.

Following the approach of Dall'Osto et al.~(2017), we cluster according
to the shape of the distribution. Prior to applying the K-means
clustering the size distribution data must therefore be normalized, and
the optimal numbers of clusters needs to be determined.

    \begin{tcolorbox}[breakable, size=fbox, boxrule=1pt, pad at break*=1mm,colback=cellbackground, colframe=cellborder]
\prompt{In}{incolor}{10}{\boxspacing}
\begin{Verbatim}[commandchars=\\\{\}]
\PY{n}{df3}\PY{p}{,} \PY{n}{df\PYZus{}daily\PYZus{}2010\PYZus{}2020\PYZus{}1h\PYZus{}mean\PYZus{}norm} \PY{o}{=} \PY{n}{fu}\PY{o}{.}\PY{n}{create\PYZus{}normalised\PYZus{}df}\PY{p}{(}\PY{n}{df\PYZus{}hourly\PYZus{}2010\PYZus{}2020\PYZus{}mean}\PY{p}{)}
\end{Verbatim}
\end{tcolorbox}

    For the hourly averaged data some endpoint columns where dropped in
order for the clustering to work. Apparently the K-means method was very
sensitive to the (in most cases) low concentrations in the endpoints.
This might indicate that the clustering method is not too robust.
Normally particle number concentrations are low in the end bins, and by
dropping them it is reasonable to assume that the overall size
distribution shape should be preserved.

    \begin{tcolorbox}[breakable, size=fbox, boxrule=1pt, pad at break*=1mm,colback=cellbackground, colframe=cellborder]
\prompt{In}{incolor}{11}{\boxspacing}
\begin{Verbatim}[commandchars=\\\{\}]
\PY{n}{df\PYZus{}hourly\PYZus{}norm\PYZus{}dropped} \PY{o}{=} \PY{n}{df\PYZus{}daily\PYZus{}2010\PYZus{}2020\PYZus{}1h\PYZus{}mean\PYZus{}norm}\PY{o}{.}\PY{n}{copy}\PY{p}{(}\PY{p}{)}
\PY{n}{df\PYZus{}hourly\PYZus{}norm\PYZus{}dropped} \PY{o}{=} \PY{n}{df\PYZus{}hourly\PYZus{}norm\PYZus{}dropped}\PY{o}{.}\PY{n}{drop} \PYZbs{}
                         \PY{p}{(}\PY{n}{columns}\PY{o}{=}\PY{p}{[}\PY{l+s+s1}{\PYZsq{}}\PY{l+s+s1}{norm5.012}\PY{l+s+s1}{\PYZsq{}}\PY{p}{,}\PY{l+s+s1}{\PYZsq{}}\PY{l+s+s1}{norm630.957}\PY{l+s+s1}{\PYZsq{}}\PY{p}{,}\PY{l+s+s1}{\PYZsq{}}\PY{l+s+s1}{norm707.946}\PY{l+s+s1}{\PYZsq{}}\PY{p}{]}\PY{p}{)}
\end{Verbatim}
\end{tcolorbox}

    \hypertarget{optimizing-the-number-of-clusters}{%
\paragraph{Optimizing the number of
clusters}\label{optimizing-the-number-of-clusters}}

    This potentially introduces some doubt whether it actually is NPF events
that the clustering captures or maybe the cluster analysis is a better
proxy for seasonal variation in the size distribution, or presence of
ultrafine particles in general. The optimal cluster number \emph{k} for
K-means clustering needs to be decided upon. This could be done using
several methods. Here the inertia method and the average silhouette
score is used.

The inertia method implies calculating the mean squared distance of each
clusters to its nearest centroid. As a rule of thumb, a good clustering
model has a low inertia and low number of clusters. If \emph{k} equals
the number datapoints, the inertia equals zero, so while choosing the
optimal \emph{k} there is a tradeoff between inertia and number of
clusters. To find the optimal \emph{k} the elbow in the inertia vs \(k\)
is identified, where the increase in \emph{k} implies a slower decrease
in inertia.

The Silouette score generates a number between -1 and 1. The closer the
Silouette score is to 1, the better cluster are distinguished from each
other. A score of 0 implies that clusters are not well distinguished and
-1 that clusters are wrongly assigned. Here, the average silouette score
is computed which informs on the average separation of all clusters.
Ideally the Silouette score is calculated for individual samples as well
(out of the scope of this work) to optimize the choice of \emph{k}.

Optimization was done for varying \emph{k} between 2 and 13 clusters.

    \begin{tcolorbox}[breakable, size=fbox, boxrule=1pt, pad at break*=1mm,colback=cellbackground, colframe=cellborder]
\prompt{In}{incolor}{12}{\boxspacing}
\begin{Verbatim}[commandchars=\\\{\}]
\PY{n}{fu}\PY{o}{.}\PY{n}{optimizeClusters}\PY{p}{(}\PY{n}{df\PYZus{}hourly\PYZus{}norm\PYZus{}dropped}\PY{p}{)}
\end{Verbatim}
\end{tcolorbox}

    \begin{center}
    \adjustimage{max size={0.9\linewidth}{0.9\paperheight}}{ReportLovisaFinal_files/ReportLovisaFinal_30_0.png}
    \end{center}
    { \hspace*{\fill} \\}
    
    \textbf{Figure 2. The inertia (left y-axis) and average silohouette
score (right y-axis) versus the number of clusters.}

    From the inertia method it is not straightforward to define the elbow,
where adding more clusters results in a minor decrease of inertia. It is
clear that the average silhoutette score of the clusters decrease with
increasing number of clusters.

The optimal cluster number was here decided to \emph{k} = 5, rather than
4 as inertia decrease less for each added cluster when \emph{k}
\textgreater{} 4. Six clusters did not improve the silouette score and
five clusters is preferred over seven as inertia is higher when \emph{k}
= 5.

    \hypertarget{method-3-utilizing-the-different-cpcs}{%
\paragraph{Method 3: Utilizing the different
CPC's}\label{method-3-utilizing-the-different-cpcs}}

The last method to identify the NPF events involves taking advantage of
the different detection limits of the UFCPC (3 nm) and CPC3010 (10 nm)
that was provided in the dataset in addition to the DMPS data. Two
different approaches where tested: taking the absolute difference
between the instruments and by computing the ratio of measured
concentrations.

    \hypertarget{results}{%
\section{Results}\label{results}}

    \hypertarget{k-means-clustering}{%
\subsection{K-means clustering}\label{k-means-clustering}}

    The clustering was performed using \(k\) = 5 clusters.

    \begin{tcolorbox}[breakable, size=fbox, boxrule=1pt, pad at break*=1mm,colback=cellbackground, colframe=cellborder]
\prompt{In}{incolor}{13}{\boxspacing}
\begin{Verbatim}[commandchars=\\\{\}]
\PY{c+c1}{\PYZsh{} Optimal cluster number }
\PY{n}{n\PYZus{}clusters} \PY{o}{=} \PY{l+m+mi}{5}

\PY{c+c1}{\PYZsh{}Perform K\PYZhy{}means clustering}
\PY{n}{silhouette\PYZus{}avg\PYZus{}1h}\PY{p}{,} \PY{n}{inertia1h}\PY{p}{,} \PY{n}{df\PYZus{}norm\PYZus{}clustered\PYZus{}1h\PYZus{}mean} \PY{o}{=} \PY{n}{fu}\PY{o}{.}\PY{n}{perform\PYZus{}clustering}\PY{p}{(}\PY{n}{df\PYZus{}hourly\PYZus{}norm\PYZus{}dropped}\PY{p}{,} \PY{n}{n\PYZus{}clusters}\PY{p}{)}

\PY{c+c1}{\PYZsh{} Define the unique number of size distribution clusters in the sence that peak diameter is different.}
\PY{n}{clusters} \PY{o}{=} \PY{n}{np}\PY{o}{.}\PY{n}{unique}\PY{p}{(}\PY{n}{df\PYZus{}norm\PYZus{}clustered\PYZus{}1h\PYZus{}mean}\PY{p}{[}\PY{l+s+s1}{\PYZsq{}}\PY{l+s+s1}{clusters}\PY{l+s+s1}{\PYZsq{}}\PY{p}{]}\PY{o}{.}\PY{n}{values}\PY{p}{)}

\PY{c+c1}{\PYZsh{} Check that the number of clusters resulting from the clustering procedure is equal to the variable ``n\PYZus{}clusters``, }
\PY{c+c1}{\PYZsh{} i.e. that the peak diameters of the clustered size distributions are unique. Print result.}
\PY{n}{fu}\PY{o}{.}\PY{n}{checkUniqueModeDiam}\PY{p}{(}\PY{n}{df\PYZus{}norm\PYZus{}clustered\PYZus{}1h\PYZus{}mean}\PY{p}{,}\PY{n}{n\PYZus{}clusters}\PY{p}{)}
\end{Verbatim}
\end{tcolorbox}

    \begin{Verbatim}[commandchars=\\\{\}]
OK! Clusters peak for different diameter when number of clusters =  5
    \end{Verbatim}

    Each hourly distribution was assigned a cluster ID during the
clustering. The mean, standard devitation, median, 10- and
90-percentiles of the clustered data was calulated for each cluster in
order to visulize the results of the clustering procedure.

    \begin{tcolorbox}[breakable, size=fbox, boxrule=1pt, pad at break*=1mm,colback=cellbackground, colframe=cellborder]
\prompt{In}{incolor}{14}{\boxspacing}
\begin{Verbatim}[commandchars=\\\{\}]
\PY{n}{df\PYZus{}norm\PYZus{}clustered\PYZus{}1h\PYZus{}mean\PYZus{}median} \PY{o}{=} \PY{n}{df\PYZus{}norm\PYZus{}clustered\PYZus{}1h\PYZus{}mean}\PY{o}{.}\PY{n}{groupby}\PY{p}{(}\PY{l+s+s1}{\PYZsq{}}\PY{l+s+s1}{clusters}\PY{l+s+s1}{\PYZsq{}}\PY{p}{)}\PY{o}{.}\PY{n}{median}\PY{p}{(}\PY{p}{)}
\PY{n}{df\PYZus{}norm\PYZus{}clustered\PYZus{}1h\PYZus{}10q}         \PY{o}{=} \PY{n}{df\PYZus{}norm\PYZus{}clustered\PYZus{}1h\PYZus{}mean}\PY{o}{.}\PY{n}{groupby}\PY{p}{(}\PY{l+s+s1}{\PYZsq{}}\PY{l+s+s1}{clusters}\PY{l+s+s1}{\PYZsq{}}\PY{p}{)}\PY{o}{.}\PY{n}{quantile}\PY{p}{(}\PY{l+m+mf}{.1}\PY{p}{)}
\PY{n}{df\PYZus{}norm\PYZus{}clustered\PYZus{}1h\PYZus{}90q}         \PY{o}{=} \PY{n}{df\PYZus{}norm\PYZus{}clustered\PYZus{}1h\PYZus{}mean}\PY{o}{.}\PY{n}{groupby}\PY{p}{(}\PY{l+s+s1}{\PYZsq{}}\PY{l+s+s1}{clusters}\PY{l+s+s1}{\PYZsq{}}\PY{p}{)}\PY{o}{.}\PY{n}{quantile}\PY{p}{(}\PY{l+m+mf}{.9}\PY{p}{)}

\PY{n}{df\PYZus{}norm\PYZus{}clustered\PYZus{}1h\PYZus{}mean\PYZus{}mean}   \PY{o}{=} \PY{n}{df\PYZus{}norm\PYZus{}clustered\PYZus{}1h\PYZus{}mean}\PY{o}{.}\PY{n}{groupby}\PY{p}{(}\PY{l+s+s1}{\PYZsq{}}\PY{l+s+s1}{clusters}\PY{l+s+s1}{\PYZsq{}}\PY{p}{)}\PY{o}{.}\PY{n}{mean}\PY{p}{(}\PY{p}{)}
\PY{n}{df\PYZus{}norm\PYZus{}clustered\PYZus{}1h\PYZus{}std}         \PY{o}{=} \PY{n}{df\PYZus{}norm\PYZus{}clustered\PYZus{}1h\PYZus{}mean}\PY{o}{.}\PY{n}{groupby}\PY{p}{(}\PY{l+s+s1}{\PYZsq{}}\PY{l+s+s1}{clusters}\PY{l+s+s1}{\PYZsq{}}\PY{p}{)}\PY{o}{.}\PY{n}{std}\PY{p}{(}\PY{p}{)}
\end{Verbatim}
\end{tcolorbox}

    Below, the mean +/- 1\(\sigma\) (shaded area) of the normalized and
clustered size distribution data is shown. The median is plotted as a
dotted line for comparison. From the graph it is clear that median and
mean are similar. From this we drew the conclusion that cluster 1 and 2
may represent NPF events, although both 1 \& 2 are bimodal and extends
to the accumulation mode. Varying the cluster number did not remove the
bimodality for the clusters whose normalized distributions peak occurs
for \(D_p\) \textless{} 100 nm. Moreover, clustering mean and median
daily distributions using 5 clusters displayed high similarity with the
clusters shown below (daily clustering is not shown here).

Comparing the results of our clustering analysis with that of Dall'Osto
et al.~(2017) and their daily mean clusters a few qualitative
observations could be made. Firstly, two of the clusters classified as
NPF indicators nucleation or bursting by Dall'Osto et al.~(2017) which
peak at 10 nm or lower, are most similar to our cluster 1 (however,
their data was not normalized, nor were clusters plotted alongside each
other which may make absolute comparison difficult).These clusters also
have tails extending into the accumulation mode although the bimodality
is absent or not as pronounced in our hourly (nor daily or median)
clusters. It should also be mentioned that their third NPF indicator
cluster (nascent) partly show similarity with our cluster 3 on the left
hand side of the peak, although our cluster 3 appear to be a
superposition of two modes and therefore we disregarded this cluster as
an indicator of NPF.

Lastly, the accumulation mode clusters (cluster 4 and 5) in our analysis
differ from the two accumulation mode clusters identified by Dall'Osto
et al.~in the sense that also display bimodality which was note the case
in their study.

    \begin{tcolorbox}[breakable, size=fbox, boxrule=1pt, pad at break*=1mm,colback=cellbackground, colframe=cellborder]
\prompt{In}{incolor}{15}{\boxspacing}
\begin{Verbatim}[commandchars=\\\{\}]
\PY{n}{fu}\PY{o}{.}\PY{n}{plotClustersNormalized}\PY{p}{(}\PY{n}{df\PYZus{}norm\PYZus{}clustered\PYZus{}1h\PYZus{}mean}\PY{p}{,} \PY{n}{diameters}\PY{p}{,}
                          \PY{n}{df\PYZus{}norm\PYZus{}clustered\PYZus{}1h\PYZus{}mean\PYZus{}mean}\PY{p}{,} \PY{n}{df\PYZus{}norm\PYZus{}clustered\PYZus{}1h\PYZus{}std}\PY{p}{,}
                          \PY{n}{df\PYZus{}norm\PYZus{}clustered\PYZus{}1h\PYZus{}mean\PYZus{}median}\PY{p}{)}
\end{Verbatim}
\end{tcolorbox}

    \begin{center}
    \adjustimage{max size={0.9\linewidth}{0.9\paperheight}}{ReportLovisaFinal_files/ReportLovisaFinal_41_0.png}
    \end{center}
    { \hspace*{\fill} \\}
    
    \textbf{Figure 3. The resulting mean (solid line) and standard deviation
(shaded) normalized sizedistribution clusters for \(k\) = 5. The median
is shown as a dotted line.}

    \hypertarget{comparison-of-npf-indicators}{%
\subsection{Comparison of NPF
indicators}\label{comparison-of-npf-indicators}}

    To make a simple comparison between the three methods used to identify
NPF events (CPC-approaches, integration method and clustering) the
annual cycle of the three proxies is shown below.

In the top panel the results from the absolute difference CPC method are
shown, and mid panel shows the difference between the CPC:s. For some
instances one of the CPC's appear not have been in operation as it
showed zero values while DMPS or other CPC was in operation. Therefore
for these occasions where one CPC equals zero, the data is excluded. In
the midpanel the integration method results for a diameter threshold
diameter of 10 nm is shown and indicated by the black dashed line, and
the shaded area shows the 10 and 90 percentiles of the data. Note that
this variable is plotted on the right y-axis. In the lower panel, the
annual cycles plotted as occurrence in hours of the clusters is shown.
The cluster chosen to represent NPF event is cluster 1 and 2 are
highlighted.

From the results in the top panel the absolute difference between CPC:s
indicate a reasonable annual NPF cycle (and similar to that of cluster
1,2 and 3), which is in contrast to the difference between the CPC:s. As
the same masking was applied to this data as for only taking the
difference between the CPCs (mid panel) the absolute difference is
likely not a good NPF proxy.

It can be observed that the data in the mid panel displays an annual
cycle not typically associated with NPF events at Zeppelin in previous
studies (Dall'Osto et al.,2017; Tunved et al., 2013). Their annual
trends indicates peak in the dark months associated with Arctic haze and
therefore this method is discarded. These indicators are therefore
disregarded as NPF indicators. Taking the ration between the UF CPC and
normal CPC to identify NPF was also discarded (not shown here) as taking
ratios between numbers with small differences makes it hard to discern
real differences. Another important conclusion is that the integration
method was not successful for the choice of thresholds equal to neither
10,20 or 30 nm (only 10 nm is shown).

Regarding the clusters in the lower panel, it can be observed that
cluster 1 and 2 exhibits annual trends expected for NPF events. However,
cluster 3 depicts a similar trend although it was discarded due its very
broad size distribution. The accumulation mode clusters (4,5) however
follow the expect annual cycle related to the Arctic Haze phenomena. It
could also be pointed out that Dall'Osto et al.~(2017) produced a
similar plot for their NPF indicating clusters indicating similar
patterns as for cluster 1-3. With regards to annual cycle (not size
distribution), our cluster 1,2,3 depicts a similar trend as to their
bursting cluster with peaks in May and close to September, rather than
in the middle of the summer which was the case for their nucleation and
nascent clusters. It can be concluded that of all studied methods, the
clustering method is likely the best proxy for further investigation of
NPF events in this project.

    \begin{tcolorbox}[breakable, size=fbox, boxrule=1pt, pad at break*=1mm,colback=cellbackground, colframe=cellborder]
\prompt{In}{incolor}{16}{\boxspacing}
\begin{Verbatim}[commandchars=\\\{\}]
\PY{c+c1}{\PYZsh{} Input DFs, clusters, threshold = 10 nm for when to integrate}
\PY{n}{fu}\PY{o}{.}\PY{n}{plotNPFproxys}\PY{p}{(}\PY{n}{df\PYZus{}hourly\PYZus{}2010\PYZus{}2020\PYZus{}mean}\PY{p}{,}
              \PY{n}{df\PYZus{}norm\PYZus{}clustered\PYZus{}1h\PYZus{}mean}\PY{p}{,}
              \PY{n}{clusters}\PY{p}{,}
              \PY{n}{bin\PYZus{}cols}\PY{p}{,}\PY{l+m+mi}{10}\PY{p}{,}\PY{n}{diameters}\PY{p}{,}\PY{n}{diameters\PYZus{}as\PYZus{}strings}\PY{p}{)}
\end{Verbatim}
\end{tcolorbox}

    \begin{center}
    \adjustimage{max size={0.9\linewidth}{0.9\paperheight}}{ReportLovisaFinal_files/ReportLovisaFinal_45_0.png}
    \end{center}
    { \hspace*{\fill} \\}
    
    \textbf{Figure 4. Top panel: The annual cysle of the absolute difference
between the UF CPC and CPC 3010. Mid panel: The left axis shows the
annual cycle of the difference between the UF CPC and CPC 3010. The
right axis shows the results of integrating up to a certain threshold
diameter (here 10 nm was chosen). Lower panel: The annual cycle of size
distribution cluster 1-5.}

    The annual frequency distribution of the clusters is displayed as a
normalized stacked bar plot of clusters over the studied years
(2010-2020). During the summer months, cluster 1,2 and 3 dominate as
opposed to the dark months where the accumulation mode clusters
dominate. This strengthens the assumption that cluster 4 and 5 are
indicators of Arctic haze and anthropogenic influence (Schmale \&
Baccarini, 2021). It is evident that cluster 1 and 2 dominate during the
summer months. Cluster 3 (not chosen to indicate NPF) appears to follow
a similar trend although it is present to higher degree than especially
cluster 1 wintertime. However, cluster 3 remains excluded as an NPF
proxy due to the shape of its size distribution.

    \begin{tcolorbox}[breakable, size=fbox, boxrule=1pt, pad at break*=1mm,colback=cellbackground, colframe=cellborder]
\prompt{In}{incolor}{17}{\boxspacing}
\begin{Verbatim}[commandchars=\\\{\}]
\PY{n}{fu}\PY{o}{.}\PY{n}{makeStackedPlot}\PY{p}{(}\PY{n}{df\PYZus{}norm\PYZus{}clustered\PYZus{}1h\PYZus{}mean}\PY{p}{,}\PY{n}{clusters}\PY{p}{)}
\end{Verbatim}
\end{tcolorbox}

    \begin{center}
    \adjustimage{max size={0.9\linewidth}{0.9\paperheight}}{ReportLovisaFinal_files/ReportLovisaFinal_48_0.png}
    \end{center}
    { \hspace*{\fill} \\}
    
    \textbf{Figure 5. The annual frequency distribution of size
distributions cluster 1-5.}

    \hypertarget{trend-for-npf-events}{%
\subsection{Trend for NPF events}\label{trend-for-npf-events}}

    From the previous results, it can be concluded that the clustering
presents the best choice in comparison to the other methods to study
trends in NPF at Zeppelin. Potentially it could be interesting to study
the trend over the years for the different clusters although such
analysis excluded taking into account variation in NPF drivers such as
meteorological conditions or variations in NPF precursors or other
processes affecting the sources of these such as sea ice extent. We
normalized the cluster frequencies to account for data coverage so that
cluster occurrence would not be biased.

First the monthly occurrence needs to be determined for each year and
cluster. \texttt{df\_clusters\_seqMonth\_T} is the transformed version
of \texttt{df\_clusters\_seqMonth} which contains one row for each
sequential month in the dataset, and columns for all the clusters where
the occurrence of each cluster is given. It also contains a column which
gives the total amount of hours a cluster was represented during this
month and the month given by number.

    \begin{tcolorbox}[breakable, size=fbox, boxrule=1pt, pad at break*=1mm,colback=cellbackground, colframe=cellborder]
\prompt{In}{incolor}{18}{\boxspacing}
\begin{Verbatim}[commandchars=\\\{\}]
\PY{n}{df\PYZus{}clusters\PYZus{}seqMonth}\PY{p}{,} \PY{n}{df\PYZus{}clusters\PYZus{}seqMonth\PYZus{}T} \PY{o}{=} \PY{n}{fu}\PY{o}{.}\PY{n}{prepareDFforTrendPlot}\PY{p}{(}\PY{n}{df\PYZus{}norm\PYZus{}clustered\PYZus{}1h\PYZus{}mean}\PY{p}{,}\PY{n}{clusters}\PY{p}{)}
\end{Verbatim}
\end{tcolorbox}

    Choose months of interest in and a dataframe which is normalized for the
data coverage for plotting is returned by \texttt{makeDFforTrend}. The
months May to July where chosen as this is the time for when the
satalitte retirieved chlorophyll has its annual peak (see Keis project).

    \begin{tcolorbox}[breakable, size=fbox, boxrule=1pt, pad at break*=1mm,colback=cellbackground, colframe=cellborder]
\prompt{In}{incolor}{19}{\boxspacing}
\begin{Verbatim}[commandchars=\\\{\}]
\PY{c+c1}{\PYZsh{} Define the period of interest in terms of months}
\PY{n}{MJJ\PYZus{}list} \PY{o}{=} \PY{p}{[}\PY{l+m+mi}{5}\PY{p}{,}\PY{l+m+mi}{6}\PY{p}{,}\PY{l+m+mi}{7}\PY{p}{]}
\PY{n}{all\PYZus{}months\PYZus{}list} \PY{o}{=} \PY{p}{[}\PY{l+m+mi}{1}\PY{p}{,}\PY{l+m+mi}{2}\PY{p}{,}\PY{l+m+mi}{3}\PY{p}{,}\PY{l+m+mi}{4}\PY{p}{,}\PY{l+m+mi}{5}\PY{p}{,}\PY{l+m+mi}{6}\PY{p}{,}\PY{l+m+mi}{7}\PY{p}{,}\PY{l+m+mi}{8}\PY{p}{,}\PY{l+m+mi}{9}\PY{p}{,}\PY{l+m+mi}{10}\PY{p}{,}\PY{l+m+mi}{11}\PY{p}{,}\PY{l+m+mi}{12}\PY{p}{]}

\PY{n}{df\PYZus{}norm\PYZus{}MJJ} \PY{o}{=} \PY{n}{fu}\PY{o}{.}\PY{n}{makeDFforTrend}\PY{p}{(}\PY{n}{df\PYZus{}clusters\PYZus{}seqMonth\PYZus{}T}\PY{p}{,}\PY{n}{MJJ\PYZus{}list}\PY{p}{)}
\PY{n}{df\PYZus{}norm\PYZus{}all} \PY{o}{=} \PY{n}{fu}\PY{o}{.}\PY{n}{makeDFforTrend}\PY{p}{(}\PY{n}{df\PYZus{}clusters\PYZus{}seqMonth\PYZus{}T}\PY{p}{,}\PY{n}{all\PYZus{}months\PYZus{}list}\PY{p}{)}
\end{Verbatim}
\end{tcolorbox}

    Below, the Theil-Sen slope and 95\% confidence interval is plotted which
indicates the trend for the months May, June and July for all clusters
between the years 2010 and 2020. The x-axis represents the sequential
months (month are counted from the first month of the data set). The
Thiel-Sen slope is a non-parametric method to discern trends. This
implies that it is robust as the method does not assume that the data is
normally distributed. It is best suited to study trends to study periods
when there is no influence of seasonality, in that case other methods
are preferred.

No obvious trend can be discerned for any of the clusters, although the
data display a varying degree of spread. The trends in satellite data
for chlorophyll and sea ice (Kei's work) were shown to be spatially
heterogenous for this period. To fully explain the results, additional
information is needed, such as airmass origin or trends in aerosol
precursors.

    \begin{tcolorbox}[breakable, size=fbox, boxrule=1pt, pad at break*=1mm,colback=cellbackground, colframe=cellborder]
\prompt{In}{incolor}{20}{\boxspacing}
\begin{Verbatim}[commandchars=\\\{\}]
\PY{n}{fu}\PY{o}{.}\PY{n}{plotThielSen}\PY{p}{(}\PY{n}{df\PYZus{}norm\PYZus{}MJJ}\PY{p}{,} \PY{n}{clusters}\PY{p}{,}\PY{l+s+s1}{\PYZsq{}}\PY{l+s+s1}{May\PYZhy{}Jun\PYZhy{}Jul}\PY{l+s+s1}{\PYZsq{}}\PY{p}{)}
\end{Verbatim}
\end{tcolorbox}

    \begin{center}
    \adjustimage{max size={0.9\linewidth}{0.9\paperheight}}{ReportLovisaFinal_files/ReportLovisaFinal_56_0.png}
    \end{center}
    { \hspace*{\fill} \\}
    
    \textbf{Figure 5. The 2010-2020 May-June-July trend for cluster 1-5. The
x-axis indicates the consequtive months starting from the first
measurement point 2010. The (non-parametric) Thiel-Sen slope is used to
estimate the trend.}

    \hypertarget{npf-relation-to-sea-ice-concentration}{%
\subsection{NPF relation to sea ice
concentration}\label{npf-relation-to-sea-ice-concentration}}

Our main aim was to make an attempt at reproducing the results from
Dall'Osto et al.~(2017) who put forward the hypothesis of an
anti-correlation between sea ice extent and NPF frequency, i.e.~that
decreasing sea ice in the Barents and Greenland Sea promotes NPF. We
investigate whether the anti-correlation holds true when more recent
data is used from the Zeppelin station. Dall'Osto et al.~(2017)
investigated the years 2000-2010 and this part of our study encompass
2010-2017 as there was an overlap between DMPS data and sea ice
satellite data for this period. Here, the annual sea ice concentration
data (Kei's work) in the Barents and Greenland Sea is used instead of
sea ice extent. We normalized the NPF observations to data coverage (in
order to remove this dependence) from cluster 1 and 2 and use the annual
sum of their frequency as those have been chosen as a proxy for NPF
events. First, the annual occurrence of cluster 1 and 2 is computed and
the sea ice concentration data is loaded.

    \begin{tcolorbox}[breakable, size=fbox, boxrule=1pt, pad at break*=1mm,colback=cellbackground, colframe=cellborder]
\prompt{In}{incolor}{21}{\boxspacing}
\begin{Verbatim}[commandchars=\\\{\}]
\PY{c+c1}{\PYZsh{} Compute the monthly and yearly normalized count for all clusters}
\PY{n}{df\PYZus{}monthly\PYZus{}cluster\PYZus{}count} \PY{o}{=} \PY{n}{df\PYZus{}norm\PYZus{}all}\PY{o}{.}\PY{n}{copy}\PY{p}{(}\PY{n}{deep} \PY{o}{=} \PY{k+kc}{True}\PY{p}{)}
\PY{n}{df\PYZus{}yearly\PYZus{}count\PYZus{}clusters} \PY{o}{=} \PY{n}{fu}\PY{o}{.}\PY{n}{DFAnnualCount}\PY{p}{(}\PY{n}{df\PYZus{}norm\PYZus{}all}\PY{p}{,}\PY{n}{clusters}\PY{p}{)}

\PY{c+c1}{\PYZsh{} Compute the total normlized count for cluster 1 and 2 by adding extra column and removing other clusters (=3,4,5)}
\PY{n}{monthly\PYZus{}cluster\PYZus{}count\PYZus{}12}\PY{p}{,} \PY{n}{df\PYZus{}yearly\PYZus{}cluster\PYZus{}count\PYZus{}12} \PY{o}{=} \PY{n}{fu}\PY{o}{.}\PY{n}{sumCluster12}\PY{p}{(}\PY{n}{df\PYZus{}monthly\PYZus{}cluster\PYZus{}count}\PY{p}{,}\PY{n}{df\PYZus{}yearly\PYZus{}count\PYZus{}clusters}\PY{p}{)}
\PY{n}{sea\PYZus{}ice\PYZus{}annual} \PY{o}{=} \PY{n}{fu}\PY{o}{.}\PY{n}{readSeaIcetoDF}\PY{p}{(}\PY{p}{)}
\end{Verbatim}
\end{tcolorbox}

    The results shown below gives little confidence in the hypothesis that
NPF at Zeppelin is anti-correlated with sea ice concentration in the
studied region between 2010-2017 (\(R^2\) = 0.21, \(p\) = 0.25). The
p-value is the two-sided p-value for a hypothesis test whose null
hypothesis is that the slope is zero (no correlation). The calculated
p-value and \(R^2\) is not enough to reject this null-hypothesis. There
are several explanations that are possible for this. For example, the
sample size is quite low, or there is simply no correlation between sea
ice concentration in the studied region and NPF, but that does not means
changes in other NPF drivers have not occurred simultaneously or in
other regions which have an effect on NPF drivers. Furthermore, it may
be that the 1h-clusters to represent NPF events chosen do not capture
the NPF events properly.

    \begin{tcolorbox}[breakable, size=fbox, boxrule=1pt, pad at break*=1mm,colback=cellbackground, colframe=cellborder]
\prompt{In}{incolor}{22}{\boxspacing}
\begin{Verbatim}[commandchars=\\\{\}]
\PY{n}{fu}\PY{o}{.}\PY{n}{plotNPFvsSeaIce}\PY{p}{(}\PY{n}{sea\PYZus{}ice\PYZus{}annual}\PY{p}{,}\PY{n}{df\PYZus{}yearly\PYZus{}cluster\PYZus{}count\PYZus{}12}\PY{p}{)}
\end{Verbatim}
\end{tcolorbox}

    \begin{center}
    \adjustimage{max size={0.9\linewidth}{0.9\paperheight}}{ReportLovisaFinal_files/ReportLovisaFinal_61_0.png}
    \end{center}
    { \hspace*{\fill} \\}
    
    \textbf{Figure 6. The 2010-2017 (normalized) total annual frequency of
cluster 1 and 2 versus sea ice concentration in the Greenland and
Barents Sea.}

    \hypertarget{discussion}{%
\section{Discussion}\label{discussion}}

NPF is an import source of CCN in the Arctic where clouds exert a net
warming effect. However, the process driving NPF, and cloud droplet
response are complex (Birch et al., 2012). The main objective of this
project was to examine if the conclusions from Dall'Osto et al.~(2017)
holds true for more recent years, i.e.~that there is a causal
relationship between sea ice loss and NPF at the Zeppelin station
located in the Arctic for. A second objective was to investigate
different methods for identification of NPF events.

To investigate the effect of sea ice melt on NPF frequency, the
identification of NPF events is crucial. Here, the K-means clustering
presented the best option in comparison to other methods that were
tested. However, using unsupervised machine learning implies little
insight to the actual clustering procedure. Two metrics were used to
evaluate the performance of the clustering, but several others are
available which might affect the choice of \(k\) which is crucial to our
analysis. The Dunn Index for example provides information how compact
and well separated clusters. The higher the Dunn Index, the better.
Dall'Osto et al.~(2017) computed the Dunn Index and average silhouette
score for their clusters, but their calculated Dunn index was close to
zero (see supplement) meaning that clusters on average are not well
separated. In this work we faced similar issues (although the Dunn Index
was not calculated) but the inertia and average silhouette score
calculation results were hard to clearly draw conclusions on the optimal
cluster number from.

Even though the clustering method may not be optimal, the results
present some interesting features. For example, the bimodality of some
of the clusters in our clustered 2010-2020 DMPS data was not as
prounounced in the work of Dall'Osto et al.~It could be speculated (if
we decide to trust our method blindly) weather the bimodality and
cluster 3 which follows the same annual trend as the NPF cluster (1 and
2), suppresses the NPF formation by providing an additional condensation
sink, and therefore we are unable to reproduce the anticorrelation of
NPF and sea ice presented by Dall'Osto et al.~(2017).

We found no evidence of an anti-correlation between NPF and sea ice
concentration. A direct comparison is impaired by the different years
Dall'Osto et al.'s study and ours target. Moreover, it is unclear
weather their NPF frequency was adjusted for data coverage. The
relationship between NPF and air mass origin is also something that
needs to be accounted for, as marine air masses that contribute to NPF
may come from regions unaffected by sea ice melt. Mixed and spatially
heterogenoues trends in chlorophyll satellite data which is indicator of
microbiological activity (and source of NPF precursor gases) emphasizes
the need for back trajectory analysis in our work.

    \hypertarget{conclusions-and-outlook}{%
\section{Conclusions and outlook}\label{conclusions-and-outlook}}

We observed no trends for individual DMPS clusters, nor an
anti-correlation of NPF to sea ice concentration in the Barents and
Greenland Sea. As Kei found that trends are spatially heterogeneous for
chlorophyll and sea ice concentration, back trajectory analysis would
provide valuable insight into the driving forces of NPF event at
Zeppelin in a rapidly warming Arctic.

For future work a more careful cluster optimization is advised. Other
features of the size distributions than shape, such as modal diameter or
width could also be extracted clustered. Another aspect of sea ice loss
is a presumed increase in anthropogenic activity in the Arctic such as
increased shipping. Looking into to changes in this over the last twenty
years could also help with the task of identifying what drives NPF at
Zeppelin.

    \hypertarget{references}{%
\section{References}\label{references}}

Birch, C. E., Brooks, I. M., Tjernström, M., Shupe, M. D., Mauritsen,
T., Sedlar, J., \ldots{} \& Leck, C. (2012). Modelling atmospheric
structure, cloud and their response to CCN in the central Arctic: ASCOS
case studies. Atmospheric Chemistry and Physics, 12(7), 3419-3435.

Dall´Osto, M., Beddows, D., Tunved, P. et al.~Arctic sea ice melt leads
to atmospheric new particle formation. Sci Rep 7, 3318 (2017).
https://doi.org/10.1038/s41598-017-03328-1

Galí, M., Devred, E., Babin, M., \& Levasseur, M. (2019). Decadal
increase in Arctic dimethylsulfide emission. Proceedings of the National
Academy of Sciences of the United States of America, 116(39),
19311--19317. https://doi.org/10.1073/pnas.1904378116

Schmale, J., \& Baccarini, A. (2021). Progress in unraveling atmospheric
new particle formation and growth across the Arctic. Geophysical
Research Letters, 48(14), e2021GL094198.

Tunved, P., Ström, J. \& Krejci, R. Arctic aerosol life cycle: linking
aerosol size distributions observed between 2000 and 2010 with air mass
transport and precipitation at Zeppelin station, Ny-Ålesund, Svalbard.
Atmos. Chem. Phys. 13, 3643--3660 (2013).

Perovich, D., Meier, W., Tschudi, M., Farrell, S., Hendricks, S.,
Gerland, S., \ldots{} \& Webster, M. (2018). Sea ice. NOAA Arctic Report
Card.

    \begin{tcolorbox}[breakable, size=fbox, boxrule=1pt, pad at break*=1mm,colback=cellbackground, colframe=cellborder]
\prompt{In}{incolor}{ }{\boxspacing}
\begin{Verbatim}[commandchars=\\\{\}]

\end{Verbatim}
\end{tcolorbox}


    % Add a bibliography block to the postdoc
    
    
    
\end{document}
